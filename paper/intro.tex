\section{Introduction}

This paper presents the derivation of an abstract machine for \lambdaPMC, a small
lazy functional language based on the pattern matching calculus
of Kahl~\cite{kahl_2004}.

Classical presentations of pattern matching in functional langugaes
define the operational semantics by translation
into case expressions with simple patterns~\cite{spj_1987,jones_1992}. For
example, consider the following Haskell function that checks whether a
list is ``short'':
\begin{verbatim}
isShort (x:y:ys) = False
isShort ys       = True
\end{verbatim}
The translation into simple case expressions is:
\begin{verbatim}
isShort xs = case xs of
                (x:xs') -> case xs' of
                              (y:ys) -> False
                              [] -> True
                [] -> True
\end{verbatim}
Nested patterns such as \verb|(x:y:ys)| must be translated into nested
case expressions with simple patterns and matches are made complete by
introducing missing constructors.  

By contrast, the translation of \textit{isShort} into \lambdaPMC\ is
straightforward:
\[ 
  \textit{isShort} = \lambda (\matchpat{(x:y:ys)}{\matchreturn{\textsf{False}}} \mid
  \matchpat{ys}{\matchreturn{\textsf{True}}}) 
\]
Compared to the version with case expressions, the \lambdaPMC\ translation
preserves a closer relation to the original source program: each
equation corresponds to one alternative in a \emph{matching
  abstraction} and nested patterns are preserved. 

Furthermore, translation into simple cases may require some
optimizations to avoid duplicating expressions whenever patterns
overlap. Consider the following example from~\cite{spj_1987} (where
$A$ and $B$ are some unspecified expressions):
\begin{verbatim}
unwieldy [] [] = A
unwieldy xs ys = B xs ys
\end{verbatim}
The translation into simple cases duplicates the sub expression $B~ xs~ ys$:
\begin{verbatim}
unwieldy xs ys = case xs of
                   [] -> case ys of
                          [] -> A
                          (y:ys') -> B xs ys
                   (x:xs') -> B xs ys
\end{verbatim}
By contrast, the translation to \lambdaPMC\ preserves the single occurrence of
the sub expression:
\[
  \textit{unwieldy} =
  \lambda\!\! \begin{array}[t]{l}
             (\matchpat{[\,]}{\matchpat{[\,]}{\matchreturn{A}}} \\
             \mid \matchpat{xs}{\matchpat{ys}}{\matchreturn{B~xs~ys}})
             \end{array}
\]
Furthermore, \lambdaPMC\ can easily handle
other extensions to pattern matching, such as as-patterns, 
pattern guards, view patterns and lambda cases.

The contributions presented in this paper are:
\begin{enumerate}
\item the definition of \lambdaPMC;
\item a big-step operational semantics for \lambdaPMC;
\item an abstract machine (i.e.\@ a small-step semantics) for \lambdaPMC;
\item a proof sketch of correctness of the abstract machine against
  the big-step semantics.
\end{enumerate}

The remaining of this paper is structured as follows:
Section~\ref{sec:syntax} defines the syntax of \lambdaPMC\ and reviews
the reduction rules of the Kahl's pattern matching
calculus. Section~\ref{sec:bigstep} defines normal forms and presents
a big-step semantics for language.  Section~\ref{sec:smallstep}
defines configurations and small-step reduction rules of an abstract
machine for \lambdaPMC\ and presents some examples.
Section~\ref{sec:soundness} presents proof sketch for correctness with
respect to the big-step semantics.
%Section~\ref{sec:experimental} presents some experimental results from a
%prototype implementation.
Section~\ref{sec:related} discusses related
work. Finally, Section~\ref{sec:conclusion} highlights
directions for further work.


% Our objective is not a more efficient implementation but rather
% to provide an operational model for lazy functional
% programs where pattern matching can be directly related to the source
% code as written by the programmer. This can be particularly useful to
% explain intensional properties, such as the interaction between
% pattern matching and evaluation demand.



%%% Local Variables:
%%% mode: latex
%%% TeX-master: "main"
%%% End:
