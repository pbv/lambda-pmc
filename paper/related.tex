
\section{Related work}

Chapter 4 of the classic textbook by Peyton Jones~\cite{spj_1987}
defines lambda abstractions with patterns $\lambda p.E$ together with
an alternative operator and a FAIL expression; the semantics for
matching is denotational and serves as the basis for defining the
correctness of compilation into case expressions.

Kahl defines a pattern matching calculus and proved its
confluence~\cite{kahl_2004}. This work forms the basis for our
\lambdaPMC\ language and the two semantics. The principle difference
is that we do not consider the ``empty expression'' corresponding to a
pattern matching failure $\lambda\matchfail$; there is simply no
reduction in such cases.

Klop et.\@ al.\@~\cite{KLOP200816} re-visit the lambda calculus with
patterns of Peyton~Jones from the point of view of combinatory
reduction systems (CRSs).

A number of researchers have looked at restricted forms of pattern
calculi.  Bucciarelli et.\@ al.\@~\cite{bucciarelli_et_al_2015} have
looked at characterizing observability for pattern matching restricted
to pairs using a type system based on intersection types.  Inspired by
this work, Alves et.\@ al.\@~\cite{alves_et_al_2020} have defined a type system based on
non-idempotent intersection types to characterize bounds on the length
of normalization sequences for such pattern
calculi.


%%% Local Variables:
%%% mode: latex
%%% TeX-master: "main"
%%% End:
