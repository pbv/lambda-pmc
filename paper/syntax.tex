\section{Syntax}\label{sec:syntax}

\begin{figure}
  \[
    \begin{array}{rcll}
      e & &     \text{expressions} \\
      e & ::= &  x  & \text{variable} \\
      & | &  \app{e_1}{e_2} & \text{function application} \\
      & | & \abstr{m} & \text{matching abstraction} \\
      & | &  \constr{c}{e_1,\ldots,e_n} & \text{constructor application} \\
      & | & \llet{\{x_i=e_i\}}{e'} & \text{let(rec) bindings} \\[2ex]
      %%
      m &  & \text{matchings} \\
      m & ::= & \matchreturn{e} & \text{return expression}\\
      & | & \matchfail & \text{failure} \\
      & | & \matchpat{p}{m} & \text{match pattern} \\
      & | & \matcharg{e}{m} & \text{argument supply}\\
      & | & \matchalt{m_1}{m_2} & \text{alternative}\\[2ex]
      %%
      p & & \text{patterns}\\
      p & ::= & x & \text{variable pattern} \\
      & | & \constr{c}{p_1,\ldots,p_n} & \text{constructor pattern} 
  \end{array}
\]
\caption{Syntax of \lambdaPMC.}\label{fig:syntax}
\end{figure}

\subsection{Expressions and matchings}
Figure~\ref{fig:syntax} defines \lambdaPMC, a small functional
language with syntactical categories for \emph{expressions},
\emph{matchings} and \emph{patterns}. Matchings and patterns are
based on the PMC calculus of Kahl~\cite{kahl_2004}.
The only extension is the \llet{\ldots}{\ldots} construct
for defining (possibly) recursive local bindings.

Matchings can be a return expression $\matchreturn{e}$ (signaling a
successful match), the matching failure \matchfail,
a pattern match $\matchpat{p}{m}$ (expecting a
single argument that must match pattern $p$), an argument application
$\matcharg{e}{m}$, or the choice \matchalt{m_1}{m_2} between two
matchings.  Note that patterns $p$ can be nested; for example
\verb|(x:(y:ys))| is a valid pattern (using infix notation for the
list constructor).

Lambda abstraction is subsumed by matching abstraction:
$\lambda x.\, e$ is equivalent to
$\lambda (\matchpat{x}{\matchreturn{e}})$.  Case expressions are also
subsumed by matching abstraction; the expression
\[
  \textsf{case}~ e_0 ~\textsf{of}~\{p_1\to e_1;\ldots; p_n\to e_n\}
\]
is equivalent to
\[
  \lambda(\matcharg{e_0}{(\matchpat{p_1}{e_1}\mid\ldots \mid\matchpat{p_n}{e_n})}) \ .
\]
%
Note that matchings in \lambdaPMC\ allow both patterns and
applications, so they may represent arbitrary expressions
and not just lambda abstractions.  


\subsection{Reduction relations}\label{sec:reduction}
%
Following~\cite{kahl_2004} we review the pattern matching calculus as
two \emph{redex} reduction relations, namely, $\expred$ between expressions and
$\matred$ between matchings. We leave the definition of an evaluation
strategy and normal forms to Section~\ref{sec:bigstep}, so this does
not yet define a complete semantics.

We use the notation $m[e'/x]$ for the substitution of free occurrences
of variable $x$ for an expression $e'$ in the matching $m$.
The definitions of free occurrences and substitution are standard and
are therefore omitted.

The first two rules state that failure is the left unit for
$\amatchalt$ while return is the left zero:
\begin{gather}
  \matchalt{\matchfail}{m} \matred m \tag{$\matchfail\amatchalt$}  \\
  \matchalt{\matchreturn{e}}{m} \matred \matchreturn{e} \tag{$\matchreturn{}\amatchalt$}
\end{gather}

The next rule states that matching abstraction built from return
expressions reduce to the underlying expression:
\begin{equation}
  \lambda \matchreturn{e} \expred e  \tag{$\lambda\matchreturn{}$} 
\end{equation}

Unlike~\cite{kahl_2004}, we do not have a reduction rule for
$\lambda \matchfail$ because we do not have an ``empty expression''
corresponding to matching failure.

Application of a matching abstraction reduces to argument supply to
the matching.  Dually, argument supply to a return expression reduces
to application of the expression.
\begin{gather}
  (\lambda m)~ a \expred \lambda(\matcharg{a}{m}) \tag{$\lambda @$}\\
  \matcharg{a}{\matchreturn{e}} \matred \matchreturn{e~a} \tag{$\amatcharg\matchreturn{}$} 
\end{gather}

The following rule propagates a matching failure through an argument supply.
\begin{equation}
  \matcharg{e}{\matchfail} \matred \matchfail \tag{$\amatcharg\matchfail$}
\end{equation}

Next, argument supply distributes through alternatives:
\begin{equation}
  \matcharg{e}{(\matchalt{m_1}{m_2})} \matred
  \matchalt{(\matcharg{e}{m_1})}{(\matcharg{e}{m_2})} \tag{$\amatcharg\amatchalt$}
\end{equation}

The following rules handle argument supply to patterns:
\begin{gather}
  \matcharg{e}{\matchpat{x}{m}} \matred m[e/x] \tag{$\amatcharg x$}\\
  \begin{split}
      \matcharg{\constr{c}{e_1,\ldots,e_n}}{&\matchpat{\constr{c}{p_1,\ldots,p_n}}{m}}
      \matred \\
      &e_1 \amatcharg p_1 \amatchpat \ldots \amatchpat e_n \amatcharg p_n \amatchpat m
    \end{split} \tag{$c\amatcharg c$}
  \\
  \matcharg{\constr{c'}{e_1,\ldots,e_k}}{\matchpat{\constr{c}{p_1,\ldots,p_n}}{m}}
  \matred
  \matchfail \tag{$c\amatcharg c'$} 
\end{gather}
In the last rule we assume $c\neq c'$.


\subsection{Examples}\label{sec:examples}
%
We will present some examples of the translation of Haskell
definitions into \lambdaPMC.  The emphasis is on illustrating the
correspondence between the Haskell source and the translated notation.

Consider the following definition of the \emph{zipWith} function that
combines two lists using a functional argument:
%
\begin{verbatim}
zipWith f (x:xs) (y:ys) = f x y : zipWith f xs ys
zipWith f xs'    ys'    = []
\end{verbatim}
%
This can be translated directly to \lambdaPMC:
%
\[
  zipWith = \lambda 
  \!\!\begin{array}[t]{l}
    (f \amatchpat (x:xs) \amatchpat (y:ys) \amatchpat \matchreturn{f~x~y : zipWith~ f~ xs~ ys}  \\
    \amatchalt f \amatchpat xs' \amatchpat ys' \amatchpat \matchreturn{[\,]})
  \end{array}
\]
%
The function argument is handled identically in both branches, so we
could factor it out:
 \[
   zipWith = \lambda (f \amatchpat 
   \!\!\begin{array}[t]{l}
     ((x:xs) \amatchpat (y:ys) \amatchpat \matchreturn{f~ x~ y : zipWith~f~xs~ys}  \\
     \amatchalt  xs' \amatchpat ys' \amatchpat \matchreturn{[\,]}))
   \end{array}
 \]

Next, consider a function \emph{nodups} which removes identical contiguous elements
from a list; this illustrates the translation of 
\emph{as-patterns} and \emph{boolean guards}:
%
\begin{verbatim}
nodups (x:xs@(y:xs')) | x==y = nodups xs
nodups (x:xs)                = x:nodups xs
nodups [] = []
\end{verbatim}
%
This can be translated to \lambdaPMC\
as a matching abstraction with three alternatives;
notice the use of a pattern argument $(x==y)$ to encode the
boolean guard~\cite{kahl_2004}:
\[
  nodups = \lambda
  \!\!\begin{array}[t]{l}
    ((x:xs) \amatchpat
    xs \amatcharg (y:xs') \amatchpat \\
     \qquad\qquad (x==y) \amatcharg \textsf{True} \amatchpat \matchreturn{nodups~ xs}  \\
    \amatchalt (x:xs) \amatchpat  \matchreturn{x: nodups~xs} \\
    \amatchalt [\,] \amatchpat \matchreturn{[\,]})
  \end{array}
\]

The translation of boolean guards generalizes
for arbitrary \emph{pattern guards}; for example:
\begin{verbatim}
addLookup env v1 v2 
  | Just r1 <- lookup env v1
  , Just r2 <- lookup env v2 = r1 + r2
\end{verbatim}
%
translates as:
%%
\[
  addLookup = \lambda \!\!\begin{array}[t]{l}
                     (env \amatchpat v1 \amatchpat v2 \amatchpat \\
                     lookup~ env~ v1 \amatcharg \cons{Just}(r1) \amatchpat \\
                     lookup~ env~ v2 \amatcharg \cons{Just}(r2) \amatchpat  \matchreturn {r1 + r2})
                     \end{array}
\]



  Pattern guards can also be used to encode
  \emph{view patterns}~\cite{wadler_1987,ghc_guide_view_patterns}
  that allow matching over abstract data types.
  Consider a Haskell data type for sequences implemented as a binary tree
  (also known as ``join lists''):
\begin{verbatim}
data JList a = Empty | Single a | Join (JList a) (JList a)
\end{verbatim}
  To allow pattern matching but keep the internal representation opaque
  we define a \emph{view type} for extracting
  the first element and a tail (a join list) and a \emph{view function}:
\begin{verbatim}
data JListView a = Nil | Cons a (JList a)
view :: JList a -> JListView a
\end{verbatim}
  Using this view function, we can define a recursive \textit{length} function
  without exposing the join list constructors:
\begin{verbatim}
length :: JList a -> Integer
length (view -> Nil) = 0
length (view -> Cons x xs) = 1 + length xs
\end{verbatim}

To match a variable \texttt{t} with a pattern \verb|view -> patt|
we need to evaluate \verb|view t| and match the result against \verb|patt|.
This can be translated into \lambdaPMC\ in a straightforward way
using pattern guards:
\[
  length = \lambda \!\!\begin{array}[t]{l}
                     (t \amatchpat view~t \amatcharg 
      \cons{Nil} \amatchpat \matchreturn {0} \\
      \mid t \amatchpat view~t \amatcharg \cons{Cons}(h,xs) \amatchpat \matchreturn{1+length~ xs})
    \end{array} \\
\]
The view function for join lists could be defined as follows; note the use
of view patterns in the view function itself:
\begin{verbatim}
view Empty = Nil
view (Single a) = Cons a Empty
view (Join (view -> Cons h t) y) = Cons h (Join t y)
view (Join (view -> Nil) y) = view y
\end{verbatim}
Again the translation in \lambdaPMC\ is straightforward:
\[
  view = \lambda\!\!
  \begin{array}[t]{l}
    (\cons{Empty} \amatchpat \matchreturn{\cons{Nil}} \\
    \mid \cons{Single}(x) \amatchpat \matchreturn{\cons{Cons}(x,\cons{Nil})} \\
    \mid \cons{Join}(xs,ys) \amatchpat view~xs \amatcharg 
    \cons{Cons}(h,t) \amatchpat \\
    \qquad \qquad\qquad\qquad \matchreturn{\cons{Cons}(h,\cons{Join}(t,ys))} \\
    \mid \cons{Join}(xs,ys) \amatchpat view~xs ~ \amatcharg
    \cons{Nil} \amatchpat \matchreturn{view~ys})
  \end{array}
\]

Finally, the GHC extensions for \emph{LambdaCase} and
\emph{LambdaCases}~\cite{ghc_guide_lambda_case} can be trivially
encoded as abstraction matchings; for example
\begin{verbatim}
sign = \case x | x>0 -> 1
               | x==0 -> 0
               | otherwise -> -1
\end{verbatim}
translates directly to:
\[ sign = \lambda (x \amatchpat \!\!\begin{array}[t]{l}
  ((x>0) \amatcharg  \textsf{True} \amatchpat \matchreturn{1} \\
  ~|~ (x==0)\amatcharg \textsf{True} \amatchpat \matchreturn{0}\\
  ~|~ \textsf{True} \amatcharg \textsf{True} \amatchpat  \matchreturn{-1} ))
 \end{array}    
\]
%
We could also avoid the trivial guard in the last alternative:
%
\[ sign = \lambda (x \amatchpat \!\!\begin{array}[t]{l}
  ((x>0) \amatcharg  \textsf{True} \amatchpat \matchreturn{1} \\
  ~|~ (x==0)\amatcharg \textsf{True} \amatchpat \matchreturn{0}\\
  ~|~  \matchreturn{-1} ))
 \end{array}    
\]
  

  % Finally, in Haskell we can make a pattern \emph{irrefutable} by prefixing it
% with a single tilde sign (\texttt{\~{}}). For example, the definition
% \begin{verbatim}
% f ~(x:xs) = rhs
% \end{verbatim}
% then matching will be delayed until variables $x$ or $xs$ are needed
% in $rhs$.  We can encode irrefutable patterns in \lambdaPMC\ as pure
% lambda abstractions applied to saturated patterns i.e.\@ essentially
% the same way as specified in the Haskell report:
% \[
%   f = \lambda(v \amatchpat
%   (\lambda(v \amatchpat {(x:xs)} \amatchpat \matchreturn{x})) \amatcharg
%   {x}  \amatchpat
%   \lambda(v \amatchpat {(x:xs)} \amatchpat \matchreturn{xs}) \amatcharg
%   {xs} \amatchpat \matchreturn {rhs})
% \]



%%% Local Variables:
%%% mode: latex
%%% TeX-master: "main"
%%% End:
