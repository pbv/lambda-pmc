\section{Natural semantics}\label{sec:bigstep}

In this section we define a natural (i.e.\@ big-step) semantics for
evaluation of expressions and matchings.  The semantics is based in
Sestof's revision~\cite{sestof_1997} of Launchbury's semantics for
lazy evaluation~\cite{launchbury_1993}.

\begin{figure}
  \[
    \begin{array}{rcll}
      e & &     \text{expressions} \\
      e & ::= &  x  & \text{variable} \\
      & | &  \app{e}{y} & \text{function application} \\
      & | & \abstr{m} & \text{matching abstraction} \\
      & | & \constr{c}{y_1,\ldots,y_n} & \text{constructor application} \\
      & | & \llet{\{x_i=e_i\}}{e'} & \text{let(rec) bindings} \\[2ex]
      %%
      m &  & \text{matchings} \\
      m & ::= & \matchreturn{e} & \text{return expression}\\
      & | & \matchfail & \text{failure} \\
      & | & \matchpat{p}{m} & \text{match pattern} \\
      & | & \matcharg{y}{m} & \text{apply argument}\\
      & | & \matchalt{m_1}{m_2} & \text{alternative}\\[2ex]
      %%
      p & & \text{patterns (as before)}
  \end{array}
\]
\caption{Normalized syntax for \lambdaPMC.}\label{fig:syntax-normalized}
\end{figure}

\subsection{Normalized syntax}
As in Launchbury's semantics, we will restrict arguments of
applications to be variables; complex arguments must be explicitly
named using let bindings.  This ensures that we can update a result
after evaluation and avoid work duplication.

Figure~\ref{fig:syntax-normalized} defines the syntax for normalized
\lambdaPMC. In the remaining of this paper we assume all expressions
and matchings to be normalized.

\subsection{Preliminary definitions}

Let \emph{heaps} $\Gamma, \Delta, \Theta$ to be finite mappings from
variables to (possibly unevaluated) expressions.  The notation
$\Gamma[\loc \mapsto e]$ in a conclusion extends a heap with an extra
entry; dually, this is also used in hypotheses to extract an entry
from a heap.

Because matchings can encode multi-argument functions and also
argument applications, we have to define the results of evaluations
accordingly. First we define a syntatical measure \arity{m} of the \emph{arity}
of a matching:
\[
  \begin{array}{rcll}
    \arity{\matchreturn{e}} &=&  \arity{\matchfail} &= 0\\
    \arity{(\matchpat{p}{m})} &=& 1 + \arity{m} \\
    \arity{(\matcharg{y}{m})} &=& \max(0, \arity{m} - 1) \\
%%    \arity{{(\matchguard{e}{\constr{c}{\vec{p}}}{m})}} &=& \arity{m} \\
     \arity{(\matchalt{m_1}{m_2})} &=& \arity{m_1} &= \arity{m_2} \\
                            & \multicolumn{2}{l}{\text{(must have equal arity)}}
  \end{array}
\]
The last condition generalizes the Haskell syntax rule that the number
of patterns for each clause of a definition must be the
same~(Section~4.4.3.1 of~\cite{haskell_2010_report}).

A \emph{weak head normal normal} (whnf) $w$ is then either
a matching abstraction of arity greater than zero or a constructor:
\[
\begin{array}{rcll}
  w  &::=&  \lambda m & \text{such that}~ \arity{m}> 0 \\
     &|& \constr{c}{y_1,\ldots,y_n}
\end{array}
\]
If $\arity{m}>0$ then $m$ expects at least one argument, i.e.\@
behaves like a lambda abstraction.  If $\arity m= 0$ then $m$ is
\emph{satured} (i.e.\@ fully applied) and therefore \emph{not} a whnf.

This definition of weak normal forms implies that a partially applied
matching will not be evaluated e.g.\@
$\lambda(\matcharg{a}{\matchpat{x}{\matchpat{y}{\matchreturn{e}}}})$
is in whnf. This is similar to a partial application in an abstract
machine with multi-argument functions e.g.\@ the STG~\cite{jones_1992}.


\begin{figure}
  \begin{gather*}
    \framebox{\ensuremath{\Gamma;\lset;e \expev \Delta; w}} \qquad \text{expression evaluation} \\[4ex]    
    %
    %
    \prooftree
    \justifies
    \Gamma; \lset; w \expev \Gamma; w
    \using{\bigrule{Whnf}}
    \endprooftree\\[3ex]
    %
    \prooftree
    \arity{m}= 0 \and
    \Gamma; \lset; [\,];\, m \matev \Delta;\matchreturn{e} \and
    \Delta; \lset; e \expev \Theta;\,w
    \justifies
    \Gamma;\lset; \lambda m \expev \Theta;w
    \using{\bigrule{Sat}}
    \endprooftree\\[3ex]
    %    %
    \prooftree
    \Gamma;\lset\cup\{\loc\}; e \expev \Delta; w 
    \justifies
    \Gamma[\loc\mapsto e]; \lset; \loc \expev \Delta[\loc\mapsto w]; w
    \using{\bigrule{Var}}
    \endprooftree \\[3ex]
    %% 
    \prooftree
    \Gamma;\lset; e\expev \Delta;\lambda m
    \and
    \Delta;\lset; \lambda (\matcharg{\loc}{m}) \expev \Theta;w 
    \justifies
    \Gamma;\lset; (e~\loc) \expev \Theta; w
    \using{\bigrule{App}}
    \endprooftree \\[3ex]
    %
    %
    \prooftree
    \Gamma[\{\loc_i\mapsto \widehat{e}_i\}];\lset; \widehat{e}' \expev \Delta;w
    \justifies
    \Gamma;\lset; \llet{\{x_i=e_i\}}{e'} \expev \Delta;w
    \using{\bigrule{Let}} 
    \endprooftree \\
    \text{where}~\loc_i ~\text{are fresh wrt}~\lset, \Gamma,e_i,e' \\
    \widehat{e}_i = e_i[\loc_1/x_1,\ldots,\loc_n/x_n], \\
    \widehat{e}' = e' [\loc_1/x_1,\ldots,\loc_n/x_n]
  \end{gather*}
  \caption{Expression evaluation}\label{fig:expr-eval}
\end{figure}

\begin{figure}
  \begin{gather*}
    %
    \framebox{\ensuremath{\Gamma;\lset;A;m\matev \Delta;\matchresult}} \qquad
    \text{matching evaluation} \\[4ex]
    %
    \prooftree
    \justifies
    \Gamma; \lset; \argstack; \matchreturn{e} \matev \Gamma; \matchreturn{\matcharg{\argstack}{e}}
    \using{\bigrule{Return}}
    \endprooftree \\[3ex] 
    %
    \prooftree
    \justifies
    \Gamma; \lset; A; \matchfail \matev \Gamma;\matchfail
    \using{\bigrule{Fail}}
    \endprooftree\\[3ex]
    %
    \prooftree
    \Gamma; \lset; (y:\argstack); m \matev \Delta;\matchresult
    \justifies
    \Gamma; \lset; \argstack; \matcharg{y}{m} \matev \Delta; \matchresult
    \using{\bigrule{Arg}}
    \endprooftree\\[3ex]
    %
    \prooftree
    \Gamma;\lset; \argstack; m[y/x] \matev \Delta; \matchresult
    \justifies
    \Gamma; \lset; (y:\argstack); \matchpat{x}{m} \matev \Delta; \matchresult
    \using{\bigrule{Bind}}
    \endprooftree \\[3ex]
    %
    \prooftree
    \begin{array}{l}
    \Gamma;\lset; y\expev \Delta; \constr{c}{y_1,\ldots,y_n} \\
    \Delta;\lset;\argstack; \matcharg{y_1}{p_1} \amatchpat \ldots \amatchpat \matcharg{y_n}{p_n} \amatchpat m
      \matev \Theta;\matchresult
    \end{array}
    \justifies
    \Gamma; \lset; (y:\argstack); \matchpat{\constr{c}{p_1,\ldots,p_n}}{m}
    \matev \Theta;\matchresult
    \using{\bigrule{Cons1}}
    \endprooftree \\[3ex]
    %
    %
    \prooftree
    \Gamma;\lset; y\expev \Delta; \constr{c'}{y_1,\ldots, y_k} \qquad
    c \neq c' \lor n \neq k
    \justifies
    \Gamma; \lset; (y:\argstack); \matchpat{\constr{c}{p_1,\ldots, p_n}}{m}
    \matev \Delta; \matchfail
    \using{\bigrule{Cons2}}
    \endprooftree    \\[3ex]
    %
    \prooftree
    \Gamma;\lset;\argstack; m_1 \matev \Delta;\matchreturn{e}
    \justifies
    \Gamma;\lset;\argstack; (\matchalt{m_1}{m_2}) \matev \Delta;\matchreturn{e}
    \using{\bigrule{Alt1}}
    \endprooftree \\[3ex]
    %
    %
    \prooftree
    \Gamma;\lset;\argstack; m_1 \matev \Delta;\matchfail  \qquad
    \Delta;\lset;\argstack; m_2 \matev \Theta; \matchresult
    \justifies
    \Gamma;\lset;\argstack; (\matchalt{m_1}{m_2}) \matev \Theta;\matchresult
    \using{\bigrule{Alt2}}
    \endprooftree
  \end{gather*}
  
  \caption{Basic matching evaluation}\label{fig:match-eval}
\end{figure}




\subsection{Evaluation rules}

Evaluation is defined in Figures~\ref{fig:expr-eval}
and~\ref{fig:match-eval} by two mutually recursive judgments:
\begin{description}
  \item[$\Gamma;\lset;e \expev \Delta;w$]  Evaluating 
   expression $e$ from heap $\Gamma$ yields heap $\Delta$ and result $w$;
  \item[$\Gamma;\lset;\argstack; m \matev \Delta;\matchresult$] 
    Evaluating matching $m$ from heap $\Gamma$ and argument stack \argstack\
    yields heap $\Delta$ and matching \matchresult.
  \end{description}

  In both judgments the set $\lset$ keeps track of variables under
  evaluation and is used to ensure freshness of variables in the
  \bigrule{Let} rule~\cite{sestof_1997}.

  In the $\matev$ judgements the argument stack \argstack\ is
  a sequence of variables representing the pending arguments
  to be applied to the matching expression.

\paragraph{Remarks about rules for  expressions (Figure~\ref{fig:expr-eval}).}
  
  Rule~\bigrule{Whnf} terminates evaluation immediately when we reach a whnf,
  namely a non-saturated matching abstraction or a constructor.
  
  Rule~\bigrule{Sat} applies to saturated matching abstractions;
  if the matching evaluation succeeds then
  we proceed to evaluate the expression returned.
    
  Rule~\bigrule{Var} forces evaluation of an expression in the heap; as in
  Launchbury and Sestof's semantics, we remove the entry
  from the heap  while performing the evaluation (``black-holing'')
  and update the heap with the result afterwards.
    
  Rule~\bigrule{App} first evaluates the function to
  obtain a matching abstraction and then evaluates
  the the argument application.
    
  Rule~\bigrule{Let} is identical to the one by Sestof: it allocates
  new expressions in the heap (taking care of renaming) and
  continues evaluating the body of the let expression.
  
  \paragraph{Remarks about rules for matchings (Figure~\ref{fig:match-eval}).}

  Rule~\bigrule{Return} terminates evaluation sucessfully
  when we reach a return expression.
  The notation $A\amatcharg e$ represents the nested applications
  of  left over arguments on the stack $A$ to the expression $e$. The definition is:
  \begin{align*}
  \matcharg{[\,]}{e} &= e \\
  \matcharg{(y:ys)}{e}  &= \matcharg{ys}{(e~y)}
  \end{align*}

  Rule~\bigrule{Fail} terminates evaluation unsuccessfully when
  we reach the match failure \matchfail.

  Rule~\bigrule{Arg} pushes an argument onto the stack and carries
  on evaluation.
  
  Rule~\bigrule{Bind} binds a variable pattern to
  an argument on the stack. This simply a renaming of the pattern
  variable $x$ to the heap variable $y$.
  
  Rules~\bigrule{Cons1} and \bigrule{Cons2} handle the successful
  and unsucessful matching of a constructor pattern; in the later
  case the matching evaluation returns \matchfail.  Note also that
  \bigrule{Cons1} continues the matching of sub-patterns in left to
  right order.
  

  Rules~\bigrule{Alt1} and \bigrule{Alt2} handle progress and failure
  in alternative matchings.
  Note that in \bigrule{Alt2} evaluation continues with the updated
  heap $\Delta$ because the evaluations effects of the failed match
  need to be shared when evaluating $m_2$.
  Note also that the argument stack is shared between $m_1$ and $m_2$
  (implementing the distributivity of $\amatcharg$ against $\mid$).
    
    
\subsection{Extension for pattern guards}\label{sec:pattern-guards}

The syntax of matchings in~Figure~\ref{fig:syntax-normalized} and
evaluation rules of Figure~\ref{fig:match-eval} deal only with
argument matches of the form $\matcharg{y}{m}$.  This prevents
encoding any interesting boolean and pattern guards: we cannot use
a let binding to name the guard expression because it will
typically refer to variables bound by the patterns.

% To encode boolean guards and pattern guards
% it is then necessary to use a let expression to name the scrutinee.
% For example, the translation of the \emph{nodups} function of
% Section~\ref{sec:examples} becomes:
% \[
%   nodups = \lambda
%   \begin{array}[t]{l}
%     (x:xs) \amatchpat
%     xs \amatcharg (y:xs') \amatchpat \\
%     \qquad \lreturn \alet ~t = (x==y) \\
%     \qquad\quad ~\ain~ \lambda (t \amatcharg \textsf{True} \amatchpat \matchreturn{nodups~ xs})\rreturn  \\
%     \amatchalt \ldots
%   \end{array}
% \]
% Operationally we can see that this will build a heap expression for $t$ that is
% immediately evaluated and never needed again.

The solution is to introduce pattern guards explicitly in the
normalized syntax for \lambdaPMC:
\[
  \begin{array}{llll}
  m &::=& \ldots & \text{(as in Figure~\ref{fig:syntax-normalized})} \\
    & | & \matcharg{e}{\matchpat{\constr{c}{\vec{p}}}{m}} & \text{pattern guards}
                                                            \end {array}
                                                          \]

The extra matching rules to deal with these special cases are  analogous to \bigrule{Cons1} and \bigrule{Cons2}:
\begin{gather*}
  \prooftree
    \begin{array}{l}
    \Gamma;\lset; e\expev \Delta; \constr{c}{y_1,\ldots,y_n} \\
    \Delta;\lset;\argstack; \matcharg{y_1}{p_1} \amatchpat \ldots \amatchpat \matcharg{y_n}{p_n} \amatchpat m
      \matev \Theta;\matchresult
    \end{array}
    \justifies
    \Gamma; \lset; \argstack; \matcharg{e}{\matchpat{\constr{c}{p_1,\ldots,p_n}}{m}}
    \matev \Theta;\matchresult
    \using{\bigrule{Guard1}}
    \endprooftree \\[3ex]
    %
    %
    \prooftree
    \Gamma;\lset; e\expev \Delta; \constr{c'}{y_1,\ldots, y_k} \qquad
    c \neq c' \lor n \neq k
    \justifies
    \Gamma; \lset; \argstack; \matcharg{e}{\matchpat{\constr{c}{p_1,\ldots, p_n}}{m}}
    \matev \Delta; \matchfail
    \using{\bigrule{Guard2}}
    \endprooftree
  \end{gather*}

  Note that if the $e$ is a single variable then these rules
  produce the same effect as the combination of \bigrule{Arg} 
  with \bigrule{Cons1} or \bigrule{Cons2}.


    

%%% Local Variables:
%%% mode: latex
%%% TeX-master: "main"
%%% End:
